\documentclass[twoside,twocolumn]{article}

\usepackage{blindtext} 
\usepackage{graphicx}
\usepackage[sc]{mathpazo} 
\usepackage[T1]{fontenc} 
\linespread{1.05} 
\usepackage{microtype} 


\usepackage[english]{babel} 


\usepackage[hmarginratio=1:1,top=32mm,columnsep=20pt]{geometry} 
\usepackage[hang, small,labelfont=bf,up,textfont=it,up]{caption} 
\usepackage{booktabs} 


\usepackage{lettrine} 


\usepackage{enumitem} 
\setlist[itemize]{noitemsep} 


\usepackage{abstract} 
\renewcommand{\abstractnamefont}{\normalfont\bfseries} 
\renewcommand{\abstracttextfont}{\normalfont\small\itshape} 


\usepackage{titlesec} 
\renewcommand\thesection{\Roman{section}} % 
\renewcommand\thesubsection{\roman{subsection}} 
\titleformat{\section}[block]{\large\scshape\centering}{\thesection.}{1em}{} 
\titleformat{\subsection}[block]{\large}{\thesubsection.}{1em}{} 


\usepackage{fancyhdr} 
\pagestyle{fancy} 
\fancyhead{} 
\fancyfoot{} 
\fancyhead[C]{Trabajo de la Unidad I: Proyecto TREVA $\bullet$ Junio 2020 $\bullet$ } 
\fancyfoot[RO,LE]{\thepage} 


\usepackage{titling} 


\usepackage{hyperref} 


%----------------------------------------------------------------------------------------
%	TILULOS
%----------------------------------------------------------------------------------------


\setlength{\droptitle}{-4\baselineskip} 

\pretitle{\begin{center}\Huge\bfseries} 
\posttitle{\end{center}} 
\title{Implementación de Call Center y asistencia para el poder judicial} 
\author{Samuel  - Franklin Carlos, Huichi Contreras, \\
Anthony Robles Flores. }
\date{\today} 
\renewcommand{\maketitlehookd}{
\begin{abstract}
\noindent 
El presente proyecto pretendió mejorar la calidad de atención de la Corte Superior de Justicia de Tacna y se enfocó en funcionalidades de asistencia y automatización de procesos y/o servicios mediante el uso de tecnologías de información. Se realizó una investigación y análisis de procesos de la institución para delimitar el alcance de esta solución. Se utilizaron herramientas como React Native, Firebase, Voximplant y DialogFlow para la construcción de la solución. Luego de implementar la propuesta, se observó un incremento en la recepción de trámites de la población de la zona y de acuerdo a una encuesta realizada, esta indicaba que un gran porcentaje de usuarios se sentía satisfecho con este canal al poder ahora recibir atención de manera casi inmediata.
\end{abstract}
\begin{abstract}
\noindent 
The present project aimed to improve the quality of attention of the Superior Court of Justice of Tacna and focused on functionalities of assistance and automation of processes and / or services through the use of information technologies. An investigation and analysis of the institution's processes was carried out to define the scope of this solution. Tools such as React Native, Firebase, Voximplant and DialogFlow were used to build the solution. After implementing the proposal, an increase in the reception of procedures from the population of the area was observed and according to a survey carried out, this indicated that a large percentage of users felt satisfied with this channel as they can now receive attention in a manner almost immediate.

\end{abstract}
}

%----------------------------------------------------------------------------------------

\begin{document}

% Print the title
\maketitle

%----------------------------------------------------------------------------------------
%	INTRODUCCION
%----------------------------------------------------------------------------------------

\section{Introduccion}
\lettrine[nindent=0em,lines=3]{A}ctualmente en el Perú las pequeñas y medianas empresas producen al mercado peruano ingresos y empleo, la gran cantidad de informacion que manejan es debido al alto numero de operaciones que realizan a diario, por lo tanto se necesita una forma de controlar los datos como las opiniones de los clientes y de esta forma conseguir retroalimentacion instantanea para las empresas. Asimismo tener la informacion en reportes descriptivos para la visualizacion se ha hecho parte importante de los sistemas de hoy para tomar desiciones acertadas y utiles para las empresasl.

\section{Titulo}
El sistema se identifica con el titulo de treva.

\section{Autores}
\begin{itemize}
\item José Edilberto, Pastor Mendoza.
\item Franklin Carlos, Huichi Contreras.
\item Sigfredo, Aponte Roldán.
\item Jesus Enrique, Sandoval blas.
\end{itemize}

\section{Planteamiento del problema}
\subsection{Problema}
La problematica general es el como las empresas evaluan el desempeño de sus servicios, y como pueden medir la satisfacción del cliente al igual de saber la efectividad y el buen manejo de sus productos.

\subsection{Justificacion}
Hoy en día los datos toman cada vez mas valor en una organizacion o empresa, de esta manera pueden asegurar una ventaja contra sus competidores y así beneficiarce para obtener una mejor calidad de servicio, mejorar su producto y por consiguiente clientes satisfechos.

\subsection{Alcance}
Para el alcance de este proyecto nesecitaremos de clientes que quieran realizar sus formularios de satisfaccion y ofrecerles todas las herramientas nesecarias para que lo hagan de manera eficaz. Alcanzado la meta podremos generar los dashboards que ayude al cliente a ver los resultados entre otros indicadores.


\section{Objetivos}
\subsection{General}
Mejorar la calidad de atención de la Corte Superior de Justicia de Tacna a la población

\subsection{Especificos}
\begin{itemize}
\item Establecer un canal de comunicación por voz automatizado que responda a las solicitudes de las personas.
\item Procesar la mayor cantidad de solicitudes.
\item Resolver consultas inmediatas que no requieran de apersonarse a la institución.
\end{itemize}

\section{Referentes teoricos}
La idea nos nacio como grupo luego de ver ejemplos de paginas como bimatico en donde manejaban estadisticas de la realizacion de las estadisticas de cada pregunta y area realizada
a continuacion pondre un ejemplos realizados esta pagina. Apartir de esas estadisticas nos dimos cuenta que pdoriamos realizar un sistema que pueda controlar todo esto desde el punto inicial hasta llegar al punto de los reportes.
\begin{figure}[h!]
	\begin{center}
		\includegraphics[width=7.5cm]{./Imagenes/esta1} 
		\caption{Estadistica de bimatico}
	\end{center}
\end{figure}
\begin{figure}[h!]
	\begin{center}
		\includegraphics[width=7.5cm]{./Imagenes/esta2} 
		\caption{Estadistica de bimatico}
	\end{center}
\end{figure}
\begin{figure}[h!]
	\begin{center}
		\includegraphics[width=7.5cm]{./Imagenes/esta3} 
		\caption{Estadistica de bimatico}
	\end{center}
\end{figure}

\section{Desarrollo de la propuesta}
Para solucionar nuestra problematica hemos desarrollado una propuesta la cual consiste en la creacion de encuestas de satifaccion mediante formularios web para los clientes usuarios de una empresa u organizacion. Para esto se creara una plataforma movil y web en la cual se podra crear, modificar y eliminar encuestas de satisfaccion personalizadas para cada empresa. El sistema evaluara esas encuestas y realizara un estudio de Inteligencia de Negocios para brindar datos de valor a la empresa u organizacion involucrada, de esta forma se busca ayudar en la toma de desiciones y mejorar la calidad de servicio.

\subsection{Tecnologia de informacion}
En esta sección definiremos las herramientas tecnologicas utilizadas para la realizacion del proyecto.
\begin{itemize}
\item Gestor de Archivos:
\subitem Github.
\item Hosting:
\subitem Hostgator.
\item Base de Datos:
\subitem MySQL.
\item Lenguajes de programación:
\subitem JavaScript.
\subitem PHP.
\item Librerias:
\subitem VueJs.
\subitem Bootstrap.
\subitem Axios.
\subitem ReactNative.
\subitem SweerAlert2.
\item Reportes:
\subitem Highchart.
\item Seguridad:
\subitem Google reCaptcha v3.
\end{itemize}

\subsection{Metodologia, tecnicas usadas}
La metodologia que usamos para la realizacion es una combinacion de Scrum y Kanban, por el lado de scrum realizamos historias de usuario y el product backlog, como tambien la division de tareas y su estimacion de importancia y tiempo. En medio del sprint usamos kanban para realizar una tabla en la cual se pondria todas las tareas que se realizarian en cada Sprint, haciendo reunionnes en cada oportunidad que tengamos para ir viendo el avanze del proyecto y su continuacion y mejora de ello. Estas tecnicas fueron escogidas por el grupo debido al tiempo que manejabamos y disponiamos en las horas de clase.

\section{Cronograma}
\begin{figure}[htb]
	\begin{center}
		\includegraphics[width=7.5cm]{./Imagenes/Cronograma} 
		\caption{Cronograma}
	\end{center}
\end{figure}


\section{Desarrollo de Solución de Mejora}

\subsection{Casos de Uso de la aplicación}
\begin{figure}[h!]
	\begin{center}
		\includegraphics[width=7.5cm]{./Imagenes/use_case} 
		\caption{Use case}
	\end{center}
\end{figure}

\subsection{Diagrama de Arquitectura de la aplicación}
Diagrama de Arquitectura de la aplicación
\begin{figure}[h!]
	\begin{center}
		\includegraphics[width=7.5cm]{./Imagenes/bd_architecture} 
		\caption{bd architecture}
	\end{center}
\end{figure}

\subsection{Diagrama de Clases de la aplicación}
Diagrama de Clases de la aplicación
\begin{figure}[h!]
	\begin{center}
		\includegraphics[width=7.5cm]{./Imagenes/diagramclass} 
		\caption{Diagram Class}
	\end{center}
\end{figure}

\section{Resultados}

Entre los hallazgos más destacables se notó que 

\section{Conclusiones}

Con los resultados expuestos observamos que el empleo oportuno de tecnologías de información como los bots y asistentes virtuales demuestran tener bastante potencial para contribuir a la mejora y alza de movimiento en los procesos de un negocio, en este caso de la Corte Superior de Justicia de Tacna.

\end{document}
